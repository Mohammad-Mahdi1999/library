\documentclass{report}

\usepackage[utf8]{inputenc}
\usepackage{geometry}
\usepackage{graphicx}
\usepackage{hyperref}
\usepackage{fancyhdr}
\usepackage[]{xepersian}
\settextfont{XB Yas.ttf} 
\geometry{a4paper, margin=1in}

% Set up fancy headers and footers
\pagestyle{fancy}
\fancyhf{} % Clear existing headers and footers
\rhead{\thepage} % Right-aligned page number
\renewcommand{\headrulewidth}{0pt} % Remove header line
\usepackage{comment}
\begin{document}
	
	\begin{titlepage}
		
		\centering
		\vspace*{2cm}
		\textbf{\LARGE امنیت شبکه}
		\vspace{1cm}
		
		
		
		\vfill
		
		\textbf{محمدمهدی فرح بخش}
		
		\vspace{1cm}
		\today
	\end{titlepage}
	
	
	\begin{center}
		\textbf{به نام خداوند بخشنده مهربان}
	\end{center}
	\section{
	فصل 1: مفاهیم اولیه
	}
	\subsection{سوال}
		\begin{enumerate}
			\item 
			امنیت چیست؟
			\item
			اقدامات امنیتی 
			\item 
			معنی لغوی secure
			\item
			منظور از
			\lr{set and forget}
			چیه؟
			\item 
			تفاوت سیاست امنیت سنتی و نوین؟
			\item
			وضعیت تعداد مهاجمان، ابزار مهاجمان،نیاز مهاجمان به دانش ، میزان نفوذ و مخارج نفوذ طی سال های اخیر چگونه است؟
			\item
			سه رکن اساسی امنیت؟
			\item 
			محرمانگی چیه، انواع و مکانیزم؟
			\item
			صحت چیه، انواع و مکانیزم امنیتی
			\item 
			دسترسی پذیری چیه، سازوکار؟
			\item
			سیاست امنیتی ؟
			\item
			آسیب پذیری ؟CVE چیه؟ دو آسیب پذیری؟
			\item 
			تهدید؟
			\item
			حمله و مهاجم؟ هر تهدید منجر به حمله میشه؟ هر حمله ای موفق نیست؟ فرق Hack و Attack؟
			\item  
			دلایل دوشواری برقراری امنیت؟
			\item
			دلایل ناامنی شبکه ها؟
			\item
			چرخه ایجاد امنیت ؟ همراه با مثال؟
			\item 
			کجای چرخه امنیتی:
			1. تله عسل،
			2. عملیات شبکه،
			3. آزمون نفوذ پذیری،
			4. دیوار آتش،
			5. سیستم های تضخیص نفوذ،
			6. رمزنگاری،
			7. آزمون نفوذ،	
			8. تصدیق هویت،
			\item 
			مصالحه ها بین امنیت و دیگر موارد؟
			\item
			انواع و دستبه بندی های حملات، هدف، نتیجه و راه های تحقق حمله؟
			\item
			خدمات امنیتی؟
			\item 
			مکانیزم رمزنگاری برای کدوم سرویس های امنیتی به کار میاد؟
			\item
			چه مکانیزمی برای سرویس امنیتی کنترل دسترسی استفاده میشه؟
			\item 
			مدل چیه؟ نیاز؟ دو مولفه؟
		\end{enumerate}
	\section{فصل 6: کد های تصدیق صحت پیام}
	
	\begin{itemize}
		\item 
		بعضی وقتا(در برخی کاربردها) صحت اهمیتش بالاتر از محرمانگی
		\item 
عملکرد ها برای تصدیق صحت پیام:
			\begin{itemize}
				 \item 
				یک تابع تولید کننده 
				$\leftarrow$
				 عامل تصدیق پیام
				 \item
				 یک تابع وارسی
				 $\leftarrow$
				 چک کردن عامل تصدیق پیام
				
			\end{itemize}
		\item 
			از الگوریتم های رمزنگاری برای تصدیق صحت پیام می شه استفاده کرد اما:
				\begin{itemize}
					\item 
					کارایی پایین
					\item
					 بررسی مفهوم بودن محتوی همواره آسان نیست
					 	\begin{itemize}
					 		\item 
					 		نیاز به قالب استاندارد
					 		\item 
					 		نیاز به افزونگی
					 		\item 
					 		دوشواری خودکار سازی فرآیند تولید و وارسی
					 	\end{itemize}
				\end{itemize}
		\item 
			هدف رمزنگاری
			$\leftarrow$
			محرمانگیست نه صحت
		\item 
			کدهای تشخیص خطا:
			\begin{itemize}
				\item \lr{Parity (CRC-1 bit)}	
					\begin{itemize}
						\item 
						تعداد 1 ها فرد بود یک دونه 1 اضافه می کنه
					\end{itemize}
				\item \lr{CRC-32 bit}
					\begin{itemize}
						\item 
						قطعات 32 بیتی رو جمع می کنه
					\end{itemize}
			\end{itemize}
		\item 
		کد تشخیص کلید ندارد
		$\leftarrow$
		برای تشخیص نویز ( غیر عمدی و غیر هوشمند) ه حمله دشمن (عمدی و هوشمند)
		
		برخلاف امضاء دو طرف قادر به ایجاد MAC هستند.
		\item 
ایراد اصلی MAC 
$\leftarrow$
 کارایی پایین
		\item 
ویژگی توابع درهم ساز:
			\begin{enumerate}
				\item 
				تابع یکطرفه
				\item 
				طول ورودی دلخواه
				\item 
				طول خروجی ثابی
				\item 
				کلید در کار نیست 
				$\leftarrow$
				برخلاف رمزنگاری و MAC
			\end{enumerate}
	
		\item 
یافتن پیام متفاوتی که به یک رشته یکسان نگاشته می شود دوشوار باشد.
			\begin{itemize}
				\item OW
				\item 2PR
				\item CR
			\end{itemize}
		\item 
با تست 

{ \large $$1.25\times2^{\frac{n}{2}}$$ }

$\leftarrow$
احتمال $50 \%$ یک تصادم پیدا می شود.
			\begin{itemize}
				\item $n$: 
				طول خروجی
			\end{itemize}
		\item 
			تابع $f$ حتما CR باشد.
		\item 
با داشتن $H(x)$ برای $x$ های نا معلوم به طول $L$
{\large $$H(x || pad(x) || L || y)$$}
			\begin{itemize}
				\item $y$:
				دلخواه
			\end{itemize}
			
			\begin{itemize}
				\item راه حل:
					\begin{itemize}
						\item
						 طول پیام قطعه اول ؟؟؟؟
						\item
						قطعه آخر با تابع $H$ متفاوت
					\end{itemize}
			\end{itemize}
	\item 

	\item 
تشابه و تفاوت MAC و Hash :
		\begin{itemize}
			\item 
			هر دو چکیده ساز 
			\item 
			کلید:
				\begin{itemize}
					\item hash
					کلید ندارد 
					$\leftarrow$
					$H(x) = y$
					\item MAC
					کلید دارد
					$\leftarrow$
					$MAC(x,Key)=y$
				\end{itemize}
		\end{itemize}
	\item 
				MD5:
				حمله روز تولد 
				$\leftarrow$
				$2^{64}$
				گام 
				$\leftarrow$
				نا امن
	\item 
				SHA:
				حمله روز تولد 
				$\leftarrow$
				$2^{80}$
				$\leftarrow$
				$2^{60.3}$
	\item 
یافتن پیام متفاوتی که به یک رشته یکسان نگاشته می شود دوشوار باشد.
	\item 
یافتن پیام متفاوتی که به یک رشته یکسان نگاشته می شود دوشوار باشد.
	\item 
یافتن پیام متفاوتی که به یک رشته یکسان نگاشته می شود دوشوار باشد.
	\end{itemize}
	\subsection{سوال}
	\begin{enumerate}
		\item 
		  آیا همیشه محرمانگی مهم است؟
		\item 
		  عملکرد های تصدیق صحت پیام کدوما هستن؟
		  \item 
		       از الگوریتم های رمزنگاری میشه استفاده کرد برای تصدیق صحت پیام؟
		  
		  \item 
		 هدف رمزنگاری چیست؟
		  \item 
		  کدهای تشخیص خطا چیا هستن؟
		  \item 
		خطای بیرونی و خطای درونی؟؟
		  \item 
  کد تشخیص خطا امنه؟ چرا؟ مثال؟
		  \item 
کد های تصدیق صحت پیام
		  \item 
توضیح MAC ؟
		  \item 
توضیح CBC-MAC ؟ حمله؟ راه حل؟ حمله؟ راه حل؟
		  \item 
تفاوت MAC با رمزنگاری؟
		  \item 
		  آیا MAC غیرقابل امضا است؟
		  \item 
روش های ترکیب MAC با رمزنگاری؟
		  \item 
ایراد اصلی MAC؟
		  \item 
ویژگی توابقع درهم ساز؟
		  \item 
امنیت توابع درهم ساز چگونه تامین میشود؟
		  \item 
حمله آزمون جامع به Hash؟
		  \item 
مرکل دمگارد MD؟ 
		  \item 
تشابه و تفاوت MAC و Hash؟
		  \item 
		  		MD5
		  چیه؟حمله؟
		  \item 
				SHA
				جیه؟حمله روز تولد؟
		  \item 
				SHA-2 
				چیه؟
		  \item 
				HMAC
				چیه؟اهداف؟
		  \item 
				مقاوم در برابر یافتن پیش نگاره اول چیه؟
		  \item 
				مقاوم در برابر یافتن پیش نگاره دوم چیه؟
		  \item 
 				کدام یک پیشنگاره اول یا پیشنگاره دوم از ویژگی مقاوم در برابر تصادم نتیجه می شود؟ چرا؟
		  \item 
				اگر تابع ویژگی << پیش نگاره دوم را داشته باشد مقاومت در برابر یافتن پیشنگاره اول را نیز دارد؟ دلیل؟
		  
	\end{enumerate}
\end{document}